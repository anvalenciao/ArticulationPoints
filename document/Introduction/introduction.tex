\section{Introducción}\label{intro}
En las ciencias de la computación, los grafos son una estructura de datos que permiten modelar diferentes tipos de problemas como, mapas (rutas de un mapa), computación distribuida (red), redes sociales (sociedad), base de datos (\textit{big data}, \textit{business intelligence}, \textit{NoSQL}) y redes neuronales.
Los puntos de articulación muestran los vértices que son críticos para mantener el grafo conectado y se pueden usar para encontrar vulnerabilidades.
El documento está organizado de la siguiente manera: Sección de la teoría de grafos, puntos de articulación, puentes y glosario de términos. Finalmente, se propone el trabajo a futuro.