% https://www.overleaf.com/learn/latex/Glossaries
% https://xlinux.nist.gov/dads/HTML/deterministicAlgorithm.html

% Búsqueda en profundidad
\newacronym{dfs}{DFS}{\textit{Depth First Search}}
\newacronym{bfs}{BFS}{\textit{Breadth First Search}}

\newglossaryentry{deterministic algorithm}
{
	name=algoritmo determinista,
	description={Su comportamiento se puede predecir completamente a partir de la entrada, el algoritmo realiza los mismos cálculos y ofrece los mismos resultados\cite{Black2009_DeterministicAlgorithm}}
}

\newglossaryentry{non-deterministic algorithm}
{
	name=algoritmo no determinista,
	description={Conceptualmente, un algoritmo con más de un paso permitido en ciertos momentos y que siempre da el paso correcto o el mejor}
}

\newglossaryentry{endpoints}
{
	name=puntos finales,
	description={Dos vértices conectados por una arista}
}

\newglossaryentry{adjacents}
{
	name=adyacentes,
	description={Si una arista conecta dos vértices, se dice que son adyacentes}
}

\newglossaryentry{incident}
{
	name=incidente,
	description={Una arista que se origina o termina en un vértice dado es incidente en ese vértice. Se dice que dos aristas que comparten un vértice son incidentes}
}

\newglossaryentry{grade}
{
	name=grado,
	description={El grado de un vértice se define como el número de aristas incidentes sobre ese vértice. Si un vértice tiene grado 0 se llama aislado. Si tiene grado 1 se llama colgante}
}

% La ruta más larga posible entre dos vértices en un grafo conectado es n-1, donde n es el número de vértices en el grafo.
% Se puede acceder a un vértice desde otro vértice si existe una ruta de cualquier longitud de uno a otro.
\newglossaryentry{geodesic}
{
	name=geodésico,
	description={La ruta más corta entre dos vértices}
}

\newglossaryentry{maximal}
{
	name=máximo,
	description={Subgrafo más grande posible}
}