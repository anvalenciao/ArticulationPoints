\begin{frontmatter}
	\title{Articulation Points} 
	\author{Andrés Valencia Oliveros\thanksref{myGitHub}\thanksref{myEmail}}
	\address{Facultad de Ingeniería, Diseño e Innovación\\ 
		Institución Universitaria Politécnico Grancolombiano\\
		Bogotá, Colombia
	}
	\thanks[myGitHub]{GitHub: 
		\href{https://github.com/anvalenciao/ArticulationPoints}{\texttt{anvalenciao}}
	}
	\thanks[myEmail]{Email: 
		\href{mailto:anvalenciao@poligran.edu.co}{
			\texttt{\normalshape anvalenciao@poligran.edu.co}
		}
	}

	\renewcommand{\abstractname}{\textbf{Resumen}}
	\begin{abstract}
		Definir si un grafo está conectado o desconectado, es un concepto fundamental de la teoría de grafos e importante en los modelos de redes, aplicables para gran variedad de problemas de decisión y que al ser analizados se pueden tomar decisiones más optimas al problema relacionado con puntos y caminos críticos.
	\end{abstract}

	\begin{keyword}
		graph, articulation point, cut vertex, bridges.
	\end{keyword}
\end{frontmatter}